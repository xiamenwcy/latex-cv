%%%%%%%%%%%%%%%%%%%%%%%%%%%%%%%%%%%%%%%%%
% "ModernCV" CV and Cover Letter
% LaTeX Template
% Version 1.1 (9/12/12)
%
% This template has been downloaded from:
% http://www.LaTeXTemplates.com
%
% Original author:
% Xavier Danaux (xdanaux@gmail.com)
%
% License:
% CC BY-NC-SA 3.0 (http://creativecommons.org/licenses/by-nc-sa/3.0/)
%
% Important note:
% This template requires the moderncv.cls and .sty files to be in the same
% directory as this .tex file. These files provide the resume style and themes
% used for structuring the document.
%
%%%%%%%%%%%%%%%%%%%%%%%%%%%%%%%%%%%%%%%%%
% 最后更新:2014年10月11日
%----------------------------------------------------------------------------------------
%   PACKAGES AND OTHER DOCUMENT CONFIGURATIONS
%----------------------------------------------------------------------------------------

\documentclass[10pt,a4paper,sans]{moderncv} % Font sizes: 10, 11, or 12; paper sizes: a4paper, letterpaper, a5paper, legalpaper, executivepaper or landscape; font families: sans or roman
% moderncv version 1.5.1 (29 Apr 2013)

\usepackage{DejaVuSansMono}
\usepackage{footmisc}
\usepackage{microtype}
\moderncvstyle{banking} % CV theme - options include: 'casual' (default), 'classic', 'oldstyle' and 'banking'
\moderncvcolor{blue} % CV color - options include: 'blue' (default), 'orange', 'green', 'red', 'purple', 'grey' and 'black'
\usepackage{pifont,amsfonts,amssymb}
\nopagenumbers{}


\usepackage[noindent]{ctex} %中文支持
\setCJKmainfont{SimSun}
% Refine quote

% Proper alignment
\usepackage[originalcommands]{ragged2e}
\renewcommand*{\cvcomputer}[4]{%
  \cvdoubleitem{#1}{\small\raggedright#2}{#3}{\small\raggedright#4}}


\newenvironment{tightitemize}
   {\begin{itemize}
   \setlength{\parskip}{3pt}}
   {\end{itemize}}
\newcommand*{\myquote}[2]{%
   \quote{\itshape #1 \\ \scshape \footnotesize #2}}


% Proper alignment
\usepackage[originalcommands]{ragged2e}
\renewcommand*{\cvcomputer}[4]{%
  \cvdoubleitem{#1}{\small\raggedright#2}{#3}{\small\raggedright#4}}

%\usepackage{lipsum} % Used for inserting dummy 'Lorem ipsum' text into the template

\usepackage[top=1cm,bottom=1cm,left=2cm,right=2cm]{geometry} % Reduce document margins
\setlength{\hintscolumnwidth}{3cm} % Uncomment to change the width of the dates column
%\setlength{\makecvtitlenamewidth}{10cm} % For the 'classic' style, uncomment to adjust the width of the space allocated to your name

%----------------------------------------------------------------------------------------
%   NAME AND CONTACT INFORMATION SECTION
%----------------------------------------------------------------------------------------
\name{王}{财勇}
% All information in this block is optional, comment out any lines you don't need

\title{\heiti  个人简历}
\address{福建省厦门市思明区厦大学生公寓}{361005}
%\address{中国科学院计算技术研究所}{海淀区, 北京市 100095}
\phone[mobile]{(+86)~158~80278326}
\email{xjwcy138@sina.cn}
%\homepage{http://blog.csdn.net/xiamentingtao}
%\social[twitter]{1442569339}
\social[github]{xiamenwcy}
\extrainfo{男,1990.11.03,籍贯:山西省介休市}

%\photo[70pt][0.4pt]{photo.jpg} % The first bracket is the picture height, the second is the thickness of the frame around the picture (0pt for no frame)
%\quote{"A witty and playful quotation" - John Smith}
%\extrainfo{Power by \LaTeX  }                 % 可选项、如不需要可删除本行

%----------------------------------------------------------------------------------------

\begin{document}

\makecvtitle % Print the CV title

%----------------------------------------------------------------------------------------
%   POSITION APPLIED(CAREER OBJECTIVE)
%----------------------------------------------------------------------------------------
\section{求职意向}
%\subsection{求职意向}

\cventry{2016年6月毕业}{意向行业:计算机,互联网,银行软件开发中心,通信}{应聘职位:研发工程师}{工作地点:广州,深圳,西安}{}{}

%----------------------------------------------------------------------------------------
%   EDUCATION SECTION
%----------------------------------------------------------------------------------------

\section{教育背景}

\cventry{2013---至今}{计算数学}{厦门大学}{硕士}{}{研究方向:计算机辅助几何设计与计算机图形学}{}  % Arguments not required can be left empty
\cventry{2009---2013}{数学与应用数学}{新疆大学}{本科}{}{}{\textit{GPA: 3.85/5}}  \\% Arguments not required can be left empty
%----------------------------------------------------------------------------------------
%   WORK EXPERIENCE SECTION
%----------------------------------------------------------------------------------------
\section{在校情况}
\cventry{}{2013.09---至今}{厦门大学}{}{}{}{
\begin{itemize}
\setlength{\itemindent}{2em}
 \item 担任班长一职,组织大家开展各项活动
 \item 参加广东移动厦大领先之星俱乐部
 \item 2014年7月初带队赴福建省漳州市东山岛开展暑期实践活动
 \item 2014年7月21日至25日,赴合肥参加中国科技大学《计算机图形学》暑期课程
 \item 2014 年12月3日至6日,赴深圳参加Siggraph Asia 2014 会议
\end{itemize}}
\section{实习经历}
\cventry{}{2013.07.01---2013.08.15}{招商银行乌鲁木齐分行实习}{}{}{}{
2013年暑假,我在招行乌鲁木齐分行友好北支行实
习,负责银行社保卡业务的办理。任职期间,负责接待社保办理个人和单位,同时与社保局接洽,确保及时地为客户办理社保卡。工作之余,我还按时做工作总结,并提交主管领导,赢得了领导的好评。实习快结束时,我总结了前面工作的经验,
为支行的社保业务设计了一套全新的工作流程,大大提升了工作的效率,并最终被评为优秀实习生。}\\
\section{项目经历}
%------------------------------------------------
\subsection{福建省数学建模与高性能科学计算重点实验室}

\cventry{2014.09---2015.04}{项目负责人}{张量积B样条曲面的自适应节点设置}{}{}{
\begin{itemize}
\setlength{\itemindent}{2em}
%   \item 设计测试方法、搭建测试平台,编写自动化脚本,完成FFmpeg/X264 编码速度测试
    \item 关键技术:三维网格的读入读出、最优化问题的L-BFGS方法求解、查询点的近似最近邻、最小二乘问题的求解
    \item 编程语言:Visual C++, Microsoft Visual Studio 2008
    \item 编程所用C++库:Openmesh、Cgal、Graphite、ANN、Eigen、libQGLViewer
    \item 该项目使用Qt4编写GUI界面,依赖OpenGL完成了三维网格的绘制和可视化操作
    \item 该项技术基于曲面的曲率等几何信息来自适应地设置节点线,从而可以更精确地拟合曲面
    \item 项目主页:{\httplink{github.com/xiamenwcy/b-spline-knots-setting}}
    \item 该项目是本人研究生以来的第一个项目,在老师的带领下,完成了项目的理论推导和C++实现,并最后撰写成论文。
\end{itemize}
}
\section{其它经历}
\cventry{2013.03---2015.1}{}{其他课程项目、研究}{}{}{}{
\begin{itemize}
\setlength{\itemindent}{2em}
 \item 使用QT4完成txt文本基本功能的实现
 \item 学习Git操作,开始用Github托管代码 {\httplink{github.com/xiamenwcy}}
 \item 学习智能优化理论,使用MATLAB优化工具箱解决一些实际问题
 \item 使用Latex编写个人简历
 \item 阅读经典著作《Effective C++》、《Effective STL》
 \item 创建个人博客{\httplink{blog.csdn.net/xiamentingtao}},同时开始关注开源项目和最新科技
 \item 学习Linux C/C++,
 \item 学习R软件,并用R软件进行数学建模与数据处理
\end{itemize}}
%----------------------------------------------------------------------------------------
%   AWARDS SECTION
%----------------------------------------------------------------------------------------

\section{荣誉奖励}
\cvitem{2009.09-2013.06}{连续四年获新疆大学基地班奖学金(每年奖励综合测评排名前1\%)}
\cvitem{2011.9}{国家奖学金}
\cvitem{2011.11}{全国大学生数学建模竞赛新疆赛区甲组三等奖}
\cvitem{2012.12}{全国大学生数学建模竞赛新疆赛区甲组二等奖}
\cvitem{2013.09}{以班级第一的成绩保送厦门大学数学科学学院}
\cvitem{2014.12}{厦门大学优秀学生干部}
\section{专业技能}
\subsection{开发}
\cvcomputer{语言}{C/C++(熟练),Qt(熟练),R(一般),MATLAB(熟练),SQL(熟练)}
           {开发环境}{Microsoft Visual Studio 2008,Qt Creator,R studio,SQL Server 2008}
\subsection{工具}
\cvcomputer{办公}{ Microsoft Office, \LaTeX}
 {操作系统}{Windows}
\subsection{证书}
\cvcomputer{英语}{ CET6(483)}
 {计算机}{计算机二级证书}
\cvcomputer{开车}{驾照C本}
 {证券}{证券从业资格证书}
\section{个人信息}
\cvitem{兴趣爱好}{\begin{itemize}
\setlength{\itemindent}{2em}
 \item  计算机,漫游互联网,写博客,接触各种电子产品
 \item  羽毛球,爬山,骑自行车,台球
 \item  喜欢抗战片、纪录片、历史片,旅游
\end{itemize}}
\cvitem{自我评价}{\begin{itemize}
\setlength{\itemindent}{2em}
 \item  踏实,努力,有上进心,有责任心。
 \item  沉稳、有耐心、意志坚定、心思缜密、做事周全。
 \item  乐于分享,喜欢Team Work。
 \item  自学能力强,勤于知识的梳理与总结。
\end{itemize}}









\end{document}
